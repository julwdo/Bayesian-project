\documentclass{Class/julia}

\usepackage{geometry}
\usepackage{graphicx} % To use \resizebox
\usepackage{array} % For custom column widths
\usepackage{calc} % To use \widthof

\usepackage{siunitx} % Formatting numbers in a table

\usepackage{amsmath}
\usepackage{subcaption}
\usepackage{threeparttable}
\usepackage{hyperref}
\usepackage{listings}
\usepackage{xcolor}
\usepackage{multirow}
%\usepackage{placeins}
\usepackage{booktabs}
\usepackage{tablefootnote}

\geometry{
    a4paper,
    total={170mm,257mm},
    left=20mm,
    top=20mm,
}

%\author{Julia Maria Wdowinska}
\date{} % Remove date from the title

\begin{document}

\begin{titlepage}
    \centering
    \vfill
    {\scshape\Large University of Milan \par}
    \vspace{0.5cm}
    {\scshape\large Faculty of Political, Economic and Social Sciences \par}
    \vspace{3cm}
    {\huge
    \textbf{A Bayesian Approach to Aggregate Insurance Claim Modeling} \\
    \vspace{0.5cm}
    \large Final Project in the Subject Bayesian Analysis \par}
    \vspace{2cm}
    {\large \textbf{Julia Maria Wdowinska} (43288A) \par}
    {\large \textbf{Edoardo Zanone} (33927A) \par}
    \vspace{0.5cm}
    {\large Data Science for Economics \par}
    {\large II Year \par}
    {\large Master’s Degree \par}
    \vfill
\begin{center}
\begin{figure}[h!]\centering
 \includegraphics[keepaspectratio=true,scale=0.2]{logo} \\
\end{figure}
\end{center}
\vfill
\begin{center}
{\small{We declare that this material, which we now submit for assessment, is entirely our own work and has not been taken from the work of others, save and to the extent that such work has been cited and acknowledged within the text of our work. We understand that plagiarism, collusion, and copying are grave and serious offences in the university and accept the penalties that would be imposed should I engage in plagiarism, collusion or copying. This assignment, or any part of it, has not been previously submitted by us or any other person for assessment on this or any other course of study.}}
\end{center}
\vfill
    {\large \today \par}
    \vfill
\end{titlepage}

\tableofcontents
%\newpage

\section{Introduction}

\section{}

The first objective of this project was to replicate the analysis conducted by Dudley. The dataset used comprises insurance claim amounts exceeding 1.5 million over a period of five years from an automobile insurance portfolio. The data, originally presented in Rytgaard (1990), is shown in Table 1.

\begin{table}[!ht]
\centering
\footnotesize
\setlength{\tabcolsep}{5pt}
\caption{Insurance Claim Amounts Exceeding 1.5 Million (Data from Rytgaard, 1990)}
\label{tab:1}
\begin{threeparttable}
\begin{tabular}{
>{\raggedright\arraybackslash}p{\widthof{Year}}
*{5}{S[table-format=2.3]}
}
\hline
\textbf{Year} & \multicolumn{5}{c}{\textbf{Claim Amounts (in millions)}} \\ \hline
\textbf{1} & 2.495 & 2.120 & 2.095 & 1.700 & 1.650 \\
\textbf{2} & 1.985 & 1.810 & 1.625 & \textendash & \textendash \\
\textbf{3} & 3.215 & 2.105 & 1.765 & 1.715 & \textendash \\
\textbf{4} & \textendash & \textendash & \textendash & \textendash & \textendash \\
\textbf{5} & 19.180 & 1.915 & 1.790 & 1.755 & \textendash \\ \hline
\end{tabular}
\begin{tablenotes}
\footnotesize
\item The threshold of 1.5 million corresponds to the retention level of an excess-of-loss insurance policy\tablefootnote{To manage risk exposure, insurers frequently employ reinsurance strategies, which help limit their financial liability on large claims. Under such arrangements, if a claim amount \( y \) exceeds a predetermined threshold \( d \) (the retention), the insurer is responsible only for paying up to \( d \), while any excess \( y - d \) is covered by the reinsurer. This approach helps insurers mitigate the impact of high-variance claims.}.
\end{tablenotes}
\end{threeparttable}
\end{table}

To model this dataset within a Bayesian framework, assumptions about the distributions of both the number of claims in year \( t \) (\( N_t \)) and the amount of the \( i \)-th claim in year \( t \) (\( Y_{i,t} \)) were necessary. Claims were assumed to occur randomly and independently at a constant rate over time, so \( N_t \) was modeled using a Poisson distribution. A Pareto distribution was chosen for \( Y_{i,t} \), as a heavy-tailed loss distribution was needed to account for the fact that individual claim amounts are positive and may include large outliers. That is,
\[
N_t \sim \text{Poisson}(\theta), \quad 0 < \theta < \infty,
\]
\[
Y_{i,t} \sim \text{Pareto}(\alpha, \beta), \quad \alpha > 0, \quad 0 < \beta < y.
\]

\noindent The \text{Pareto}\((\alpha, \beta)\) distribution with support \( [\beta, \infty) \) was particularly suitable in this context, as we were modeling claim amounts exceeding a certain threshold.

In addition, the following assumptions were made:

\begin{itemize}
\item \( N_t \) are independently and identically distributed (i.i.d.) across \( t \),
\item \( Y_{i,t} \) are i.i.d.\ across both \( i \) and \( t \),
\item \( N_t \) and \( Y_{i,t} \) are independent for all \( i \) and \( t \).
\end{itemize}

\noindent Under these assumptions, the aggregate claim amount in year \( t \), denoted by
\[
S_t = Y_{1,t} + Y_{2,t} + \cdots + Y_{N_t,t},
\]
follows a compound Poisson distribution, since it represents the sum of independent Pareto-distributed random variables.







\end{document}