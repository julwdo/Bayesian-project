\documentclass{Class/julia}

\usepackage{geometry}
\usepackage{graphicx} % To use \resizebox
\usepackage{array} % For custom column widths
\usepackage{calc} % To use \widthof

\usepackage{siunitx} % Formatting numbers in a table

\usepackage{amsmath}
\usepackage{subcaption}
\usepackage{threeparttable}
\usepackage{hyperref}
\usepackage{listings}
\usepackage{xcolor}
\usepackage{multirow}
%\usepackage{placeins}
\usepackage{booktabs}

\geometry{
    a4paper,
    total={170mm,257mm},
    left=20mm,
    top=20mm,
}

%\author{Julia Maria Wdowinska}
\date{} % Remove date from the title

\begin{document}

\begin{titlepage}
    \centering
    \vfill
    {\scshape\Large University of Milan \par}
    \vspace{0.5cm}
    {\scshape\large Faculty of Political, Economic and Social Sciences \par}
    \vspace{3cm}
    {\huge
    \textbf{A Bayesian Approach to Aggregate Insurance Claim Modeling} \\
    \vspace{0.5cm}
    \large Final Project in the Subject Bayesian Analysis \par}
    \vspace{2cm}
    {\large \textbf{Julia Maria Wdowinska} (43288A) \par}
    {\large \textbf{Edoardo Zanone} (33927A) \par}
    \vspace{0.5cm}
    {\large Data Science for Economics \par}
    {\large II Year \par}
    {\large Master’s Degree \par}
    \vfill
\begin{center}
\begin{figure}[h!]\centering
 \includegraphics[keepaspectratio=true,scale=0.2]{logo} \\
\end{figure}
\end{center}
\vfill
\begin{center}
{\small{We declare that this material, which we now submit for assessment, is entirely our own work and has not been taken from the work of others, save and to the extent that such work has been cited and acknowledged within the text of our work. We understand that plagiarism, collusion, and copying are grave and serious offences in the university and accept the penalties that would be imposed should I engage in plagiarism, collusion or copying. This assignment, or any part of it, has not been previously submitted by us or any other person for assessment on this or any other course of study.}}
\end{center}
\vfill
    {\large \today \par}
    \vfill
\end{titlepage}

\tableofcontents
%\newpage

\section{Introduction}

\section{}

The first objective of this project was to replicate the analysis conducted by Dudley. The dataset used comprises insurance claim amounts exceeding 1.5 million over a period of five years from an automobile insurance portfolio. The data, originally presented in Rytgaard (1990), is shown in Table 1.

\begin{table}[!ht]
\centering
\footnotesize
\setlength{\tabcolsep}{5pt}
\caption{Insurance Claim Amounts Exceeding 1.5 Million (Data from Rytgaard, 1990)}
\label{tab:1}
\begin{threeparttable}
\begin{tabular}{
>{\raggedright\arraybackslash}p{\widthof{Year}}
*{5}{S[table-format=2.3]}
}
\hline
\textbf{Year} & \multicolumn{5}{c}{\textbf{Claim Amounts (in millions}} \\ \hline
\textbf{1} & 2.495 & 2.120 & 2.095 & 1.700 & 1.650 \\
\textbf{2} & 1.985 & 1.810 & 1.625 & \textendash & \textendash \\
\textbf{3} & 3.215 & 2.105 & 1.765 & 1.715 & \textendash \\
\textbf{4} & \textendash & \textendash & \textendash & \textendash & \textendash \\
\textbf{5} & 19.180 & 1.915 & 1.790 & 1.755 & \textendash \\ \hline
\end{tabular}
\begin{tablenotes}
\footnotesize
\item Note:\ The threshold of 1.5 million corresponds to the retention level of an excess-of-loss insurance policy\footnote{To manage risk exposure, insurers frequently employ reinsurance strategies, which help limit their financial liability on large claims. Under such arrangements, if a claim amount \( y \) exceeds a predetermined threshold \( d \) (the retention), the insurer is responsible only for paying up to \( d \), while any excess \( y - d \) is covered by the reinsurer. This approach helps insurers mitigate the impact of high-variance claims. A related practice involves using deductibles to avoid processing numerous small claims. Since handling minor claims can be disproportionately expensive, insurers may impose a deductible---requiring policyholders to cover costs up to \( d \). In some cases, a franchise deductible is used instead: here, the insurer pays the entire claim \( y \) only if it exceeds the threshold \( d \).}.
\end{tablenotes}
\end{threeparttable}
\end{table}

To model this data, the following mathematical notation was introduced. Let \( N_t \) denote the number of claims in year \( t \), and let \( Y_{i,t} \) represent the amount of the \( i \)-th claim in year \( t \), where \( i = 1, 2, \ldots, N_t \). The aggregate claim amount in year \( t \) is then given by

\[
S_t = Y_{1,t} + Y_{2,t} + \cdots + Y_{N_t,t}.
\]

To proceed, certain assumptions about the distributions of \( N_t \) and \( Y_{i,t} \) were necessary. It was assumed that claims occur at a constant rate over time at random. Therefore, the number of claims \( N_t \) was modeled using a Poisson distribution. Since the individual claim amounts \( Y_{i,t} \) are positive and may include large outliers, a loss distribution with a heavy tail was needed. Consequently, the Pareto distribution was selected to model \( Y_{i,t} \). Hence,

\[
N_t \sim \text{Poisson}(\theta), \quad 0 < \theta < \infty,
\]
\[
Y_{i,t} \sim \text{Pareto}(\alpha, \beta), \quad \alpha > 0, \quad 0 < \beta < y.
\]

In the Pareto distribution, the parameter \( \beta \) determines the minimum possible value of \( y \), meaning the support of the distribution is \( [\beta, \infty) \). This is particularly appropriate in our context, as we are modeling insurance claims that exceed a specific threshold.

Additional assumptions were made: \( N_t \) was assumed to be independently and identically distributed (i.i.d.) across \( t \); \( Y_{i,t} \) were i.i.d. across both indices \( i \) and \( t \); and \( N_t \) and \( Y_{i,t} \) were assumed to be independent for all \( i \) and \( t \). Under these assumptions, \( S_t \) follows a compound Poisson distribution, since it represents the sum of independent Pareto-distributed random variables.



\end{document}